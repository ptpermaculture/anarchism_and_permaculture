%        File: permaculture_and_anarchism.tex
%     Created: Sat Apr 28 03:00 PM 2012 W
% Last Change: Sat Apr 28 03:00 PM 2012 W
%
\documentclass[a4paper, 11pt]{article}
\usepackage{marginnote}
\usepackage[]{color}
\usepackage{multicol}
\usepackage{marginnote}
\renewcommand*{\marginfont}{\color{red}}
\edef\marginnotetextwidth{\the\textwidth}
\usepackage [inner=4cm, outer=4cm, twoside] {geometry}
\usepackage{verbatim}
\begin{document}
\title{Permaculture and Anarchism}
\author{Harry Wykman}
\date{14th of December, 2012}
\maketitle
\begin{multicols*}{2}

\section*{Introduction}
Life within ecological limits implies anarchism. Conversely anarchism possesses the shape required for ecological care.  These statements seems to me succinct ways to express the most fundamental relationship between permaculture and anarchism. These two traditions, with their distinct and intertwined histories, philosophies and cultures, appear to possess a fundamental harmony.  I would like to begin to bring into conversation and to rejoin certain threads of these two colourful garments which are both so often misunderstood.  For me this conversation is not of mere historical or theoretical interest --- it is my conviction that the traditions and practicioners represented by permaculture and anarchism will be vital as we move into an uncertain future.

Anarchism is concerned with liberatory forms of organisation. In this exploration we will be most concerned with that part of the anarchist tradition which works against all forms of domination, especially economic domination. Permaculture, however variously it may have evolved, is a genetic descendant of this portion of anarchist thought.

Permaculture is about design.  Design in permaculture is directed towards using plentiful and renewable energy sources, principally solar energy through green plants, to create places in which people can live within regional ecological limits while producing an abundance to meet their needs.

\section*{Anarchist Protopermaculturalists}
\marginnote[left]{\small{\textsc{Kropotkin} (1842 -- 1921)}} Many libertarian and anarchist thinkers anticipated the ideas which would become crystalised within permaculture.  In this way they can be considered proto-permaculturalists. \textsc{David Holmgren}, co-originator of the concept of permaculture, was `strongly influenced' by the work of \textsc{Peter Kropotkin} as he developed the cluster of ideas which came to be known as permaculture.  \textsc{Kropotkin} gave up an early but already distinguished career in geography to devote his significant energies to writing and organising for anarchist causes. Through his writings he challenged social Darwinian notions of competition as the sole driver of evolutionary processes and showed that cooperation or mutual aid was as much or more important as a factor in evolution.  \textsc{Kropotkin} also wrote books and pamphlets sketching the kind of new society he thought was just around the corner which integrated the best of scientific advances such as widespread electricity, labour saving technology and intensive agriculture with worker self-management and federerated communes.  \textsc{Kropotkin}'s ideas --- that cooperation is evolutionarily successful and that there is a possible synthesis of the best of industrial and pre-industrial life --- are foundational to permaculture.

\marginnote{\small{\textsc{Fourier} (1772 -- 1837) and the \textsc{Nihilists}}}  Permaculture's ties with the broad libertarian tradition are closer, however.  \textsc{Kropotkin} was amongst the most prominent of the Russian \textsc{Nihilists}. Many of the \textsc{Nihilists} had been influenced by the ideas of \textsc{Charles Fourier}, an early libertarian thinker. \textsc{Fourier} writes that: 

\vspace{1mm}
``\em{Nature seeks to reduce the time given to factory work as much as possible by organising life in such a manner that all products are brought to perfection.  According to this principle, factories will not be concentrated as they are today in cities choked with swarms of wretched creatures.  Rather, they will be scattered throughout all the world's rural areas and communities.}\em''
\vspace{1mm}

\textsc{Fourier}'s vision of ruralised industry was combined with a critique of large-scale agriculture and promotion of an intensive, diverse garden agriculture with associated free workers and communities.  The Russian nihilists saw this Fourierist program as especially possible for Russia, which remained relatively rural compared with the industrialised centres of Europe.  \textsc{Fourier}'s visions of industry integrated with rural life, of a diverse intensive agriculture and of organisation for convivial work and fair distribution of resources anticipate the kinds of ideas and ethical drivers which form a part of the breadth of permaculture.

\marginnote{\small{\textsc{\'Elis\'e Reclus} (1830 -- 1905)}}\textsc{\'Elis\'ee Reclus} was, like \textsc{Kropotkin}, a geographer.  Having experimented with cooperatives in different forms prior to the ferment of the Paris Commune in 1872, \textsc{Reclus} then wrote prolifically on the possibilities of a harmony between the human species and the places it inhabits in his nineteen volume \em{Universal Geography}\em.  He writes:

\vspace{1mm}
``\em{Man who loves the earth knows that the issue is to preserve it, also to increase its beauty and to give back what has been taken from it by sheer brutal exploitation.  Aware that his own interest is blended with the interest of others, he repairs the damages committed by his predecessors and he helps the earth, rather than brutally assaulting it, and works for the beautification as well as betterment of his environment.  He knows, not only as an agriculturalist and industrialist, to make better use of the products and forces of the globe but he also learns, as an artist, to give to the land that surrounds him more charm, grace and majesty; he knows to realise \em{the landscapes suggested by the painters}\em.  Becoming ``the consciousness of the earth'' man assumes, by virtue of it, a responsibility to be in harmony with the surrounding nature.}\em''
\vspace{1mm}

If we can leave aside the patriarchal language of this passage, we can discern several threads of thought here which come very close to the some of the basic assumptions of permaculture and the motivations of the permaculture activist.  It is summed up by the word `regeneration'.\footnote{`Regeneraci\'on' was, fittingly, the title of a mexican anarchist newspaper edited by \textsc{Ricardo Flores Mag\`on} at the turn of the 20$^{th}$ century}  Anarchism knows both the social and ecological possibilities of regeneration. Like permaculture, the anarchism of Reclus is founded upon the idea that we, \em{homo sapiens}\em, as a part of nature and through understanding the possibilities of human creativity and intellect, might come to be co-creators, with nature, of continuously improving and dynamically stable natural systems.  The harmony that \textsc{Reclus} seeks is the same harmony that permaculture seeks --- a harmony which knows the solar powered abundance of natural systems and, through exercise of intellect and creativity in each particular and unique place of the earth, comes to use the sun and other abundant energy sources as the foundation for long-term and socially just habitation of the places of the earth.

\section*{Shapes Become Possible}
We have begun historically for several reasons.  Firstly to draw out the similarities between permaculture and anarchism over relatively recent history.  Additionally, to show that permaculture and anarchism share an approach to history and to the futures that might grow from history.  Both identify tendencies and ways of being in the past as somehow important for their present and future shape.  Both seek a synthesis of these past ways of being with liberatory modes of organisation and technology only available in the present.  In this way many of the organisational shapes anarchists have imagined in the past are now more possible than ever.  In \em{Fragments of an Anarchist Anthropology}\em, \textsc{David Graeber} writes that:

\vspace{1mm}
``\em{The nineteenth century `founding fathers' did not think of themselves as having invented anything particularly new.  The basic principles of anarchism --- self-organisation, voluntary association, mutual aid --- referred to forms of human behaviour they assumed to have been around as long as humanity.}\em''
\vspace{1mm}

Anarchism is continous with past human impulses toward freedom even though the term itself is recent.

Nevertheless, anarchists, as we have already seen, were not advocating a return to some past way of being.  Similarly, permaculture surveys the past from the view afforded by the energy peak of industrial society to identify ecologically successful peoples.  Success here is evidenced by ecological health over time.  Both anarchism and permaculture are critical of industrial society but both also embrace the possibilities for an harmonious synthesis of pre-industrial and industrial modes of life for a socially just and ecologically regenerative future.  For example, the kind of low-overhead, small scale industry with diverse outputs combined with agriculture envisioned by \textsc{Kropotkin} in \em{Fields, Factories and Workshops of Tomorrow }\em has become more possible than ever with advances in precision computer controlled machine tools.  This synthesis is perhaps best represented in the present by projects like \textsc{Open Source Ecology} --- a project for the development of a set of machines for low-energy village scale development utilising high technology.

\section*{Permaculture as Revolutionary Action}

When anarchists talk about revolution, most anarchists are not talking about seizing the mechanisms of power.  The \textsc{Zapatistas} exemplify one anarchist practice of revolution --- the opening up of resistant, liberated and liberatory spaces within the cracks of the current system.  In the case of the \textsc{Zapatistas} they first rended wider those cracks through militant action.  \textsc{David Graeber} defines revolutionary action as:

\vspace{1mm}
``\em{\ldots any collective action which rejects, and therefore confronts, some form of power or domination and in doing so, reconstitutes social relations --- even within the collectivity --- in that light.}\em'' --- \em{Fragments of An Anarchist Anthropology}\em
\vspace{1mm}

The best applications of permaculture are just such revolutionary action.  The design-thinking that permaculture represents and the set of strategies which are associated with it are tools for resisting the ecologically destructive forces of capitalism.  Resistance in this case takes the form of viable domestic and regional economies.

\section*{A Necessary Shape}

We began this exploration with the statement that `anarchism possesses the shape required for ecological care.' This is because ``Care of the earth,'' one of the three ethics of permaculture distilled from a typology of ecologically successful groups, has certain requirements of scale, organisation, knowledge and culture.  Care of the earth is here intended to mean the practical care of the many patches of earth, not merely the abstract care of the planet.  It is not possible to care in this practical way for the earth's many places without the sensitivity and knowledge born of living closely with a patch of land within a particular bioregion.  As \textsc{Murray Bookchin} has written:

\vspace{1mm}
``\em{If food cultivation is to be a mode of cooperation with nature rather than a contest between opponents, the agriculturist must become thoroughly familiar with the ecology of the land; he must acquire a new sensitivity to its needs and possibilities. This presupposes the reduction of agriculture to a human scale, the restoration of moderate-sized agricultural units, and the diversification of the agricultural situation; in short, it presupposes a decentralized, ecological system of food cultivation.}\em'' --- \em{Post Scarcity Anarchism}\em
\vspace{1mm}

Permaculture implies the kind of decentralised society envisioned by the anarchist.  In the same way, the decentralisation and federation born from anarchism's desire for liberatory relationships between persons forms the shape for a more ecologically sound mode of being.  This is not to suggest that the ecological concerns are somehow irrelevant in anarchist organisation, because they are not.  \textsc{Graham Purchase} concludes his essay on \em{Anarchism and Ecological Thought}\em in this way:

\vspace{1mm}
``\em{As a matter of historical fact, anarchism, unlike any other political and social philosophy, has practically and theoretically supported the concepts of the ecological region, alternative energy, green cities and smaller-scale organic farming techniques for more than two centuries.  Historic links between anarchism and ecological thought are no accident --- anarchists didn't simply stumble upon the correct practical solutions to our burgeoning ecological crisis.  Rather anarchism's conception of a future and more ecologically-integrated social existence was and is based upon a profound, well thought-out and deeply cherished anarchist life philosophy containing important ecological insights based upon the rational and scientific observation of natural life processes.}\em'' 
\vspace{1mm}

Just as anarchism is the only modern political philosophy with an historical depth of ecological thinking, so also permaculture with its shorter history is not simply concerned with `nature' --- ``Care of people'' and ``Set Limits to Consumption and Reproduction and Redistribute Surplus'' are also foundational ethics of permaculture.  Where permaculture has privileged ecological strategies and anarchism social ones these two traditions have much to share with one another.

\section*{Towards An Harmonious Future}

``\em{\ldots anarchism is a \em prefigurative \em practice \ldots }\em'' --- `Anarchist Economics: A Holistic View' in \em{The Accumulation of Freedom}\em
\vspace{1mm}

Both permaculture and anarchism are prefigurative --- they are about a partial imagining of a better future and getting down to growing the seeds of that future today. Part of permaculture's strength, like the Occupy movement, is that is draws together people with diverse approaches to politics.  I hope that I have shown that the shape of permaculture is essentially political.  It is my belief that permaculture practice will be required to become more and more political as we confront the drawn-out death throws of capitalism in the midst of ecological crisis.  The future that permaculture activists and anarchists might imagine together will be better designed and more capable of solidly resisting attempts to coopt or destroy it than either group or tradition of thought is capable of alone.  The direct action of the permaculture activist and the anarchist is one that will be capable of resistance and sustenance.  Both will be required for a regenerative future.

\begin{comment}
\begin{quote}
%BELLO and BAVIERA in Magdoff and Tokar - 'peasants' and small farmers as model
\begin{quote}
        ``As the capitalist mode of production enters its worst crisis since the 1930s, peasant and small farmers increasingly present a vision of autonomy, diversity, and cooperation that may just be the key elements of a necessary social and economic reorganisation.'' --- \textsc{Bello and Baviera} --- Food Wars in \textsc{Magdoff and Tokar} --- Agriculture and Food Crisis, p49
\end{quote}

%CHOMSKY - limits of design - 'observe and interact'
\begin{quote}
    ``Nobody is smart enough to plan a society.  You can talk about some of the principles upon which a society should work, and you can set up guidelines as to how to implement them, and how to experiment with them, and there are probably many different ways of doing them.  There's no reason to believe that there's only one right answer; there are lots of different answers, with advantages and disadvantages, and people have to choose between them on the basis of experience, what has happened to others, and so on.  This is true in every area.'' \textsc{Noam Chomsky} in \textsc{Pannekoek}, Workers' Councils p xi
\end{quote}

%BOOKCHIN - ecologically destructive nature of capitalism
    ``Ecologically, bourgeois exploitation and manipulation are undermining the very capacity of the earth to sustain advanced forms of life.'' --- \textsc{Bookchin}, PSA p58
\end{quote}

\begin{quote}
%BOOKCHIN - necessity of human manipulation - sensitivity in agriculture - implications for shape of food production
    ``The need for a productive agriculture—itself a form of interference with nature—must always remain in the foreground of an ecological approach to food cultivation and forest management. No less important is the fact that man can often produce changes in an ecosystem that would vastly improve its ecological quality. But these efforts require insight and understanding, not the exercise of brute power and manipulation. This concept of management, this new regard for the importance of spontaneity, has far-reaching applications for technology and community—indeed, for the social image of man in a liberated society. It challenges the capitalist ideal of agriculture as a factory operation, organized around immense, centrally controlled land-holdings, highly specialized forms of monoculture, the reduction of the terrain to a factory floor, the substitution of chemical for organic processes, the use of gang-labor, etc. \textbf{If food cultivation is to be a mode of cooperation with nature rather than a contest between opponents, the agriculturist must become thoroughly familiar with the ecology of the land}; he must acquire a new sensitivity to its needs and possibilities. \textbf{This presupposes the reduction of agriculture to a human scale, the restoration of moderate-sized agricultural units, and the diversification of the agricultural situation; in short, it presupposes a decentralized, ecological system of food cultivation}.'' --- \textsc{Bookchin}, PSA p64
\end{quote}

\begin{quote} 
%BOOKCHIN - ecocommunity - utopian ideal and natural reality - comprehensible social scale - common focus of ecology and idealism
    ``The implications of small-scale agriculture and industry for a community are obvious: if humanity is to use the principles needed to manage an ecosystem, the basic communal unit of social life must itself become an ecosystem — an ecocommunity. It too must become diversified, balanced and well-rounded. By no means is this concept of community motivated exclusively by the need for a lasting balance between man and the natural world; it also accords with the Utopian ideal of the rounded man, the individual whose sensibilities, range of experience and lifestyle are nourished by a wide range of stimuli, by a diversity of activities, and by a social scale that always remains within the comprehension of a single human being. Thus the means and conditions of survival become the means and conditions of life; need becomes desire and desire becomes need. The point is reached where the greatest social decomposition provides the source of the highest form of social integration, bringing the most pressing ecological necessities into a common focus with the highest Utopian ideals.'' --- \textsc{Bookchin}, PSA p65
\end{quote}

\begin{quote}
%PANNEKOEK - technology and the shape of communism
    ``It is precisely the more advanced form of capitalism, with its advanced technology, high productivity, and network of communications, which offers a material base for the establishment of communism based on a system of workers' councils.'' --- \textsc{Paul Mattick} in \textsc{Pannekoek}, Workers' Councils p xxxiii
\end{quote}
\begin{quote}
%PANNEKOEK - the advantages of mutual aid in nature
    ``[The Bourgeois Darwinists] claimed that only the extermination of all the weak is in accordance with nature and that it is necessary to prevent the deterioration of the race, while protection of the weak is unnatural and leads to degeneration. But what do we see? In nature itself, in the animal world, we find that the weak are protected, that they don't need to persist by their individual strength, and that they are not exterminated due to their individual weakness. And this arrangement does not weaken a group in which it is the rule, but strengthens it. The animal groups in which mutual aid is best developed maintain themselves best in the struggle for existence.'' --- Anton Pannekoek, Darwinisme en Marxisme (in Dutch), 1909.
\end{quote}

\begin{quote}
%HOLMGREN - influence of Kropotkin - Kropotkin on cooperation in nature
    ``Charles Darwin’s emphasis on competitive and predatory relationships in driving evolution was based on some excellent observations of wild nature, but he was also influenced by his observations of the society around him. Early industrial England was a rapidly changing society tapping new energy sources. Predatory and competitive economic relationships were overturning previous social norms and conventions. The social Darwinists used Darwin’s work to explain and justify industrial capitalism and the free market. Peter Kropotkin was one of the first ecological critics of the social Darwinists. He provided extensive evidence from both nature and human history that co-operative and symbiotic relationships were at least as important as competition and predation. Kropotkin’s work had a strong influence on my early thinking in developing the permaculture concept. See P. Kropotkin, Mutual Aid, 1902.'' --- \textsc{David Holmgren}, \textit{The Essence of Permaculture}, 2007
\end{quote}
\begin{quote}
%KINNA - local communities gaining own wellbeing
    ``\ldots social transformation relies upon the ability of individuals working in communities to find ways of securing their own sources of well-being: food, shelter, and clothing.'' --- Ruth Kinna, Preface to The Accumulation of Freedom: Writings on Anarchist Economics, 2012, AK Press, p7
\end{quote}
\begin{quote}
SHANNON, NOCELLA AND ASKIMAKOPOULOS - prefiguration
``\ldots anarchism is a \textem{prefigurative} practice \ldots '' --- ``Anarchist Economics: A Holistic View'' in tAoF, 2012, AK Press, p12
\end{quote}
\begin{quote}
%NETTLAU - anarchism, economics and ethics
    ``To argue economically without ethics has served us nothing.'' --- Christiaan Cornelissen in Netlau, A Short History of Anarchism, pxxii 
\end{comment}
%\begin{comment}
%\end{comment}
\end{multicols*}
\end{document}

