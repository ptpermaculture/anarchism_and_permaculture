%        File: permaculture_and_anarchism.tex
%     Created: Sat Apr 28 03:00 PM 2012 W
% Last Change: Sat Apr 28 03:00 PM 2012 W
%
\documentclass[a4paper, 11pt]{article}
\usepackage{marginnote}
\usepackage[]{color}

\renewcommand*{\marginfont}{\color{red}}
\edef\marginnotetextwidth{\the\textwidth}
\usepackage [inner=4cm, bindingoffset=1cm, twoside] {geometry}
\usepackage{verbatim}
\begin{document}
\title{Permaculture and Anarchism}
\author{Harry Wykman}
\date{28th of April, 2012}
\maketitle

\section{Introduction}
Both anarchism and permaculture resist easy definition. Both are frequently misunderstood.  Misunderstanding is often deliberately propagated, at other times it derives from the actions of representatives of each tradition and sometimes from the internal diversity of the traditions.  In this essay I would like to bring into conversation or to rejoin certain threads of these two colourful garments which might be important as we move into an uncertain the future.

\marginnote[left]{\small{Anarchism and Permaculture}} The outline of the part of the anarchist tradition which most interest us in this exploration surrounds that portion which works against all forms of domination, especially economic domination.  Permaculture, however variously it may have evolved, is arguably a genetic descendant of anarchist thought described by that outline.

Permaculture is about design.  Design in permaculture is directed towards using plentiful and renewable energy sources, principally solar energy through green plants, to create places in which people can live within regional ecological limits while producing an abundance to meet their needs.
\begin{comment}
\begin{quote}
%BELLO and BAVIERA in Magdoff and Tokar - 'peasants' and small farmers as model
\begin{quote}
        ``As the capitalist mode of production enters its worst crisis since the 1930s, peasant and small farmers increasingly present a vision of autonomy, diversity, and cooperation that may just be the key elements of a necessary social and economic reorganisation.'' --- \textsc{Bello and Baviera} --- Food Wars in \textsc{Magdoff and Tokar} --- Agriculture and Food Crisis, p49
\end{quote}

%CHOMSKY - limits of design - 'observe and interact'
\begin{quote}
    ``Nobody is smart enough to plan a society.  You can talk about some of the principles upon which a society should work, and you can set up guidelines as to how to implement them, and how to experiment with them, and there are probably many different ways of doing them.  There's no reason to believe that there's only one right answer; there are lots of different answers, with advantages and disadvantages, and people have to choose between them on the basis of experience, what has happened to others, and so on.  This is true in every area.'' \textsc{Noam Chomsky} in \textsc{Pannekoek}, Workers' Councils p xi
\end{quote}

%BOOKCHIN - ecologically destructive nature of capitalism
    ``Ecologically, bourgeois exploitation and manipulation are undermining the very capacity of the earth to sustain advanced forms of life.'' --- \textsc{Bookchin}, PSA p58
\end{quote}

\begin{quote}
%BOOKCHIN - necessity of human manipulation - sensitivity in agriculture - implications for shape of food production
    ``The need for a productive agriculture—itself a form of interference with nature—must always remain in the foreground of an ecological approach to food cultivation and forest management. No less important is the fact that man can often produce changes in an ecosystem that would vastly improve its ecological quality. But these efforts require insight and understanding, not the exercise of brute power and manipulation. This concept of management, this new regard for the importance of spontaneity, has far-reaching applications for technology and community—indeed, for the social image of man in a liberated society. It challenges the capitalist ideal of agriculture as a factory operation, organized around immense, centrally controlled land-holdings, highly specialized forms of monoculture, the reduction of the terrain to a factory floor, the substitution of chemical for organic processes, the use of gang-labor, etc. \textbf{If food cultivation is to be a mode of cooperation with nature rather than a contest between opponents, the agriculturist must become thoroughly familiar with the ecology of the land}; he must acquire a new sensitivity to its needs and possibilities. \textbf{This presupposes the reduction of agriculture to a human scale, the restoration of moderate-sized agricultural units, and the diversification of the agricultural situation; in short, it presupposes a decentralized, ecological system of food cultivation}.'' --- \textsc{Bookchin}, PSA p64
\end{quote}

\begin{quote} 
%BOOKCHIN - ecocommunity - utopian ideal and natural reality - comprehensible social scale - common focus of ecology and idealism
    ``The implications of small-scale agriculture and industry for a community are obvious: if humanity is to use the principles needed to manage an ecosystem, the basic communal unit of social life must itself become an ecosystem — an ecocommunity. It too must become diversified, balanced and well-rounded. By no means is this concept of community motivated exclusively by the need for a lasting balance between man and the natural world; it also accords with the Utopian ideal of the rounded man, the individual whose sensibilities, range of experience and lifestyle are nourished by a wide range of stimuli, by a diversity of activities, and by a social scale that always remains within the comprehension of a single human being. Thus the means and conditions of survival become the means and conditions of life; need becomes desire and desire becomes need. The point is reached where the greatest social decomposition provides the source of the highest form of social integration, bringing the most pressing ecological necessities into a common focus with the highest Utopian ideals.'' --- \textsc{Bookchin}, PSA p65
\end{quote}

\begin{quote}
%PANNEKOEK - technology and the shape of communism
    ``It is precisely the more advanced form of capitalism, with its advanced technology, high productivity, and network of communications, which offers a material base for the establishment of communism based on a system of workers' councils.'' --- \textsc{Paul Mattick} in \textsc{Pannekoek}, Workers' Councils p xxxiii
\end{quote}
\begin{quote}
%PANNEKOEK - the advantages of mutual aid in nature
    ``[The Bourgeois Darwinists] claimed that only the extermination of all the weak is in accordance with nature and that it is necessary to prevent the deterioration of the race, while protection of the weak is unnatural and leads to degeneration. But what do we see? In nature itself, in the animal world, we find that the weak are protected, that they don't need to persist by their individual strength, and that they are not exterminated due to their individual weakness. And this arrangement does not weaken a group in which it is the rule, but strengthens it. The animal groups in which mutual aid is best developed maintain themselves best in the struggle for existence.'' --- Anton Pannekoek, Darwinisme en Marxisme (in Dutch), 1909.
\end{quote}

\begin{quote}
%HOLMGREN - influence of Kropotkin - Kropotkin on cooperation in nature
    ``Charles Darwin’s emphasis on competitive and predatory relationships in driving evolution was based on some excellent observations of wild nature, but he was also influenced by his observations of the society around him. Early industrial England was a rapidly changing society tapping new energy sources. Predatory and competitive economic relationships were overturning previous social norms and conventions. The social Darwinists used Darwin’s work to explain and justify industrial capitalism and the free market. Peter Kropotkin was one of the first ecological critics of the social Darwinists. He provided extensive evidence from both nature and human history that co-operative and symbiotic relationships were at least as important as competition and predation. Kropotkin’s work had a strong influence on my early thinking in developing the permaculture concept. See P. Kropotkin, Mutual Aid, 1902.'' --- \textsc{David Holmgren}, \textit{The Essence of Permaculture}, 2007
\end{quote}
\begin{quote}
%KINNA - local communities gaining own wellbeing
    ``\ldots social transformation relies upon the ability of individuals working in communities to find ways of securing their own sources of well-being: food, shelter, and clothing.'' --- Ruth Kinna, Preface to The Accumulation of Freedom: Writings on Anarchist Economics, 2012, AK Press, p7
\end{quote}
\begin{quote}
SHANNON, NOCELLA AND ASKIMAKOPOULOS - prefiguration
``\ldots anarchism is a \textem{prefigurative} practice \ldots '' --- ``Anarchist Economics: A Holistic View'' in tAoF, 2012, AK Press, p12
\end{quote}
\begin{quote}
%NETTLAU - anarchism, economics and ethics
    ``To argue economically without ethics has served us nothing.'' --- Christiaan Cornelissen in Netlau, A Short History of Anarchism, pxxii 
\end{comment}
%\begin{comment}
\section{OUTLINE NOTES for article: Permaculture and Anarchism}
\begin{enumerate}
    \item - Introduction
    \item - Anarchism and Permaculture
    \item - A Necessary Shape
        \begin{itemize}
            \item - the requrements of ecosystems (sensitivity)
            \item - Berry
            \item - permaculture and community
        \end{itemize}
    \item - A Shape Become Possible
        \begin{itemize}
            \item - the neotechnic
            \item - machine tools (despite origins)
            \item - Mattick quote
            \item - OSE and Swaraj
        \end{itemize}
\end{enumerate}
%\end{comment}
\end{document}

\end{document}
